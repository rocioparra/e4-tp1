\documentclass[e4_tp1_main.tex]{subfiles}

\begin{document}

\section{Ejercicio 2}


El circuito que analizaremos ahora es una fuente buck, es decir, un convertidor DC/DC donde la tensi\'on de salida es menor que la de entrada. El mismo puede observarse en la figura \ref{fig:buck-ideal}. En esta secci\'on, consideraremos ideal a la llave con la que se hace el switching.


\begin{figure}[ht]
	\centering
	
	\includegraphics[width=0.7\textwidth]{images/ej2/buck_ideal.pdf}
	\caption{Fuente buck con llave ideal}
	\label{fig:buck-ideal}
\end{figure}


Esta fuente debe cumplir los siguientes requerimientos:

\begin{table}[ht]
	\centering
	\begin{tabular}{c|c|c}
		$V_I$ (V) & $V_O$ (V) & $\nicefrac{\Delta V_O}{V_O}_{max}$ (\%) \\ \hline
		9.0       & 3.7       & 5                                       \\ 
	\end{tabular}
	\caption{Requerimientos de entrada/salida de la fuente}
	\label{tab:vin-vout}
\end{table}


Esto se debe lograr a una frecuencia de switching de $f_s = 50$kHz. En cuanto a los componentes pasivos, los reactivos son sugeridos por la c\'atedra: $L=220\upmu$H y $C=10\upmu$F. La resistencia de carga  debe ser elegida de manera tal que, en primera instancia, la fuente funcione en modo continuo.



\subsection{An\'alisis te\'orico}

\subsubsection{Con componentes ideales}

Para obtener la salida indicada, debemos seleccionar el duty adecuado. Esto se puede obtener planteando que en r\'egimen permanente, $\langle V_L \rangle = 0$, y por lo tanto, $\langle V_O \rangle = \langle V_D \rangle$. Considerando al diodo como ideal, su tensi\'on es 0 cuando la llave est\'a abierta, y $V_I$ cuando est\'a cerrada. Por lo tanto, despejado para $D$ obtenemos que:
\begin{equation}
	D = \frac{V_O}{V_I} \simeq 0.41
\end{equation}  

Con este valor, podemos ahora obtener la corriente de boundary. Sabiendo que cuando la llave est\'a abierta, $V_L = L \frac{di_L}{dt} = -V_0$, y que esta condici\'on se mantiene por un tiempo $T_s \cdot (1-D)$, se obtiene:
\begin{equation}
	\Delta I_L = \frac{V_O}{L} \cdot (1-D) \cdot T_s \simeq 0.20\text{A}
\end{equation} 

Con lo cual la corriente de boundary es:
\begin{equation}
	I_{B} = \frac{\Delta I_L}{2} \simeq 0.10\text{A}
\end{equation}

Para que $I_O > I_B$, elegimos pues $R = 10\Omega$, lo cual resulta en una corriente de salida de 0.37A.

El ripple de tensi\'on, por otro lado, es entonces de:
\begin{equation}
	\frac{\Delta V_O}{V_O} 
	= \frac{1}{V_O} \cdot \frac{\Delta Q}{C} 
	= \frac{1}{V_O} 
	\cdot \frac{1}{C} \cdot \frac{1}{2} \cdot \frac{\Delta I_L}{2} \frac{T_s}{2}
	\simeq 1.23 \%
\end{equation}

Este valor se encuentra por debajo del m\'aximo aceptable de 5\%.


\subsubsection{Considerando la tensi\'on forward del diodo}

El an\'alisis anterior sirve como primera aproximaci\'on del comportamiento del circuito. Sin embargo, a la hora de simular, resulta evidente que no es suficiente: la tensi\'on obtenida a la salida es considerablemente menor a la que se requiere, de alrededor de 3.2V.

En primer lugar, podemos observar que si bien es cierto que $V_O = \langle V_D \rangle$, en la secci\'on anterior consideramos que cuando la llave est\'a cerrada, la tensi\'on en el diodo es nula. Sin embargo, sabemos que esto no es cierto: el diodo estar\'a forward-biased, con lo cual su tensi\'on no ser\'a otra que la de forward. 

En la datasheet del 
MUR460\footnote{
	\url{https://www.onsemi.com/pub/Collateral/MUR420-D.PDF}
}, el diodo sugerido por la c\'atedra, la figura 6 muestra la relaci\'on entre la tensi\'on forward y la corriente forward. Como la corriente ser\'a la de salida, 0.37A, la tensi\'on estar\'a entre $\sim$0.75V (a 100$^\circ$C) y 
$\sim$0.85V (a 25$^\circ$C). 
De la figura 9, obtenemos que para una onda cuadrada de corriente forward de 0.37A, la potencia disipada ser\'a de alrededor de 0.4W, que teniendo en cuenta que $R_{\Theta JA}=50$ (nota 2 de la datasheet, asumiendo montaje 1), el diodo calentar\'a alrededor de 20$^\circ$C sobre la temperatura ambiente. Por lo tanto, la temperatura no ser\'a ni 25 ni 100 grados, si no que rondar\'a los 40, dependiendo de la temperatura ambiente. Supondremos pues $V_{FD}= 0.8$V de tensi\'on forward en el diodo.

Una vez que contamos con este valor, podemos calcular el nuevo valor de la tensi\'on de salida: 
\begin{equation}
	V_O = \langle V_D \rangle = D \cdot V_I - (1-D) \cdot V_{DF}   
\end{equation}

Despejando para $D$, obtenemos:
\begin{equation}
	D = \frac{V_O + V_{DF}}{V_I + V_{DF}} 
	= \frac{3.7V + 0.8V}{9V + 0.8V} 
	\simeq 0.46 
\end{equation}

Esto a su vez cambiar\'a el valor de los ripples de tensi\'on y corriente, puesto que no s\'olo a la tensi\'on de la bobina durante $T_{off}$ se le suma la tensi\'on forward del diodo, sino que adem\'as al aumentar $D$, disminuye $T_{off}$. Resulta entonces: 
\begin{equation}
	\Delta I_L = \frac{V_O + V_{DF}}{L} \cdot (1-D) \cdot T_s
	\simeq 0.22\text{A}
\end{equation} 

Con este valor, la corriente de boundary sube a 0.11A, con lo cual a\'un seguimos operando en modo continuo con 10$\Omega$ de carga. En cuanto al ripple de tensi\'on:
\begin{equation}
\frac{\Delta V_O}{V_O}  
= \frac{1}{V_O} 
\cdot \frac{1}{C} \cdot \frac{1}{2} \cdot \frac{\Delta I_L}{2} \frac{T_s}{2}
\simeq 1.53 \%
\end{equation}


\subsubsection{Considerando las ESR de la bobina y el capacitor}

Si tenemos en cuenta las ESR, el circuito queda con la configuraci\'on que se observa en la figura \ref{fig:buck-esrs}. Para seguir cumpliendo con $\langle V_L \rangle = 0$, por lo tanto ahora 

\begin{figure}[ht]
	\centering
	\includegraphics[width=0.7\textwidth]{images/ej2/buck_esrs.pdf}
	\caption{Fuente buck, considerando las ESR de la bobina y del capacitor}
	\label{fig:buck-esrs}
\end{figure}

\end{document}


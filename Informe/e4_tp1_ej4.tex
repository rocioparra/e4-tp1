\documentclass[e4_tp1_main.tex]{subfiles}

\begin{document}

\section{Ejercicio 4}

\subsection{Modo discontinuo}
En esta sección se muestran las diversas curvas ya calculadas en las secciones anteriores pero trabajando en modo de conducción discontinua (DCM). Para ello, es necesario calcular la corriente de boundary para saber el valor de corriente a partir del cual se comenzará a trabajar en modo discontinuo.

\subsubsection{Señal de disparo}

\subsubsection{Conmutación de la llave}

\subsubsection{Tensión en el inductor}

\subsubsection{Corriente en el inductor}

\subsubsection{Corriente en el diodo}


\subsection{Pérdidas en modo continuo y discontinuo}

\end{document}

\begin{equation}
	x=\frac{1}{3}+V_{o}^{2}
\end{equation}	
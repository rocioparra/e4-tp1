% ! Tex root = e4_tp1_main

\usepackage[utf8]{inputenc}
% caracteres utf8 (tildes, enie) sin tener que usar comandos

\usepackage[spanish, es-tabla, es-nodecimaldot]{babel} 
% texto automatico en espaniol
% "tabla" en vez de "cuadro"
% no reemplaza puntos decimales por comas

%% NO AGREGAR PAQUETES ANTES DE ESTO, ES IMPORTANTE QUE BABEL ESTE PRIMERO

%%%%%%%%%%%%%%%%%%%%%%%%%%%%%%%%%
%% PAQUETES EXTRA %%%%%%%%%%%%%%%
%%%%%%%%%%%%%%%%%%%%%%%%%%%%%%%%%

\usepackage{hyperref}					% Hyperlinks on pdf (must be called before Geometry)

\usepackage[a4paper, total={7in, 9in}, footskip=50px]{geometry} 

\usepackage{subfiles}
\usepackage{xr} % permite referencias a labels de archivos externos con \externaldocument{filename.tex}

\usepackage{amsmath} % PAQUETES DE MATEMATICA
\usepackage{amsfonts}
\usepackage{amssymb}


\usepackage{booktabs} % tablas lindas

\usepackage{upgreek} % letras griegas no italic
\usepackage{steinmetz} % comando \phase{}
\usepackage{units} % permite usar nicefrac
\usepackage{graphicx} % importar imagenes
\usepackage[space]{grffile} % permite tener espacios en paths de imagenes
\usepackage{float} % posicion H para floats
\usepackage[colorinlistoftodos]{todonotes}
\usepackage{wrapfig} % figuras wrappeadas por texto
%\usepackage{floatrow}
\usepackage{afterpage} % permite hacer clear de floats sin page break 



% margenes correctos en subarchivos

\setlength{\parindent}{10pt}			%cuanta sangria al principio de un parrafo
\usepackage{indentfirst}				%pone sangria al primer parrafo de una seccion
\usepackage{subcaption}
\usepackage{lscape}

% Header style
\usepackage{fancyhdr}
\setlength{\headheight}{15.2pt}
\pagestyle{fancy}
\lhead{22.14 Electr\'onica 4}
\chead{TP1: Convertidores DC/DC}
\rhead{Grupo 2}
\cfoot{\thepage}




\hypersetup{
	colorlinks=true,
	linkcolor=blue,
	filecolor=magenta,      
	urlcolor=blue,
	citecolor=blue,    
}